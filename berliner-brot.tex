\recipe{Berliner Brot} {
	\begin{category}[Guss]
		\ingredient{150}{g}{Puderzucker}
		\ingredient{2}{EL}{Wasser}
	\end{category}
	\begin{category}[Teig]
		\ingredient{3}{}{Eier}
		\ingredient{375}{g}{Zucker}
		\ingredient{100}{g}{Apfelkraut}
		\ingredient{50}{g}{Zitronat}
		\ingredient{1}{flacher TL}{Nelkenpfeffer}
		\ingredient{1}{gehäufter TL}{Zimt}
		\ingredient{2}{EL}{Rum}
		\ingredient{100}{g}{Zartbitterschokolade}
		\ingredient{400}{g}{Mehl}
		\ingredient{1}{gehäufter TL}{Backpulver}
		\ingredient{200}{g}{Haselnüsse}
		\ingredient{100}{g}{abgezogene und \\ halbierte Mandeln}
	\end{category}
} {
	\step[Teig]{
		Eier und Wasser verrühren.
		Zucker zugeben und alles schaumig rühren.
		Die übrigen Zutaten (außer Nüsse und Mandeln) unterrühren.
		Mehl und Backpulver zugeben und verrühren.
		Mit dem Löffel Nüsse und Mandeln unterheben.
	}
	\step[Backen]{
		Das Ganze auf ein mit Backpapier ausgelegtes Blech streichen und bei ca. 180 – 200$^\circ$C Ober/Unterhitze ca. 25 - 30 Minuten backen.
	}
	\step[Guss]{
		Puderzucker mit Wasser anrühren.
		Direkt nach dem Backen das Berliner Brot mit Guss bestreichen und möglichst in warmem Zustand schneiden.
	}
	\hints{
		Mandeln werden abgezogen, indem man sie ein paar Minuten in kochendes Wasser legt, dann das Wasser abschütten, und die Haut einfach abziehen.
		Mandeln halbieren geht am besten auf einem Brettchen, d.h. jede einzeln zwischen Daumen und Zeigefinger nehmen und dann durchschneiden.	
	}
}