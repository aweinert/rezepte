\recipe{Big-Mac-Rolle}{
	\begin{category}[Teig]
		\ingredient{250}{g}{Magerquark}
		\ingredient{100}{g}{geriebener Käse}
		\ingredient{3}{}{Eier}
	\end{category}
	\begin{category}[Soße]
		\ingredient{2}{TL}{Ketchup}
		\ingredient{1}{TL}{Senf}
		\ingredient{2}{EL}{Naturjoghurt}
	\end{category}
	\begin{category}[Füllung]
		\ingredient{250}{g}{Hackfleisch}
		\ingredient{3}{Scheiben}{Schmelzkäse}
		\ingredient{3}{}{saure Gurken}
		\ingredient{}{}{Salat, Tomaten}
		\ingredient{}{}{Salz, Pfeffer}
	\end{category}
} {
	\step[Teig]{Magerquark, Reibekäse und Eier zu einer dickflüssigen Masse verrühren und auf einem mit Backpapier ausgelegten Backblech verteilen. Bei 180 Grad ca. 20 Minuten backen. Den Teig auskühlen lassen.}
	\step[Soße]{	Die Zutaten für die Sauce verrühren.}
	\step[Füllung]{Das Hackfleisch in einer Pfanne mit Salz und Pfeffer krümelig anbraten. Die Gurken in Scheiben schneiden und zum Hackfleisch geben. 2/3 der Sauce auf dem gebackenen Teig verteilen. Das noch warme Hackfleisch darüber verteilen und den Toastkäse über dem Hackfleisch schmelzen lassen.}
	\step{Anschließend Salat und evtl. Tomaten darauf legen und die restliche Sauce darauf verteilen. Den Teig mithilfe des Backpapiers einrollen, so dass die Big Mac Rolle aussieht wie eine Biskuitrolle.}
	\hints{Wichtig ist, nicht zu viel auf dem Teig zu verteilen, da die Rolle sonst zu dick wird und die Füllung an beiden Enden hervorquillt.}
}
